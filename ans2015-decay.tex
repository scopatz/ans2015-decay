\documentclass{anstrans}
%%%%%%%%%%%%%%%%%%%%%%%%%%%%%%%%%%%
\title{Binary Formulation of Decay Equations}
\author{Anthony M. Scopatz,$^{*}$ Paul P. H. Wilson$^{*}$}

\institute{
$^{*}$The University of Wisconsin-Madison, 1500 Engineering Drive, Madison,
WI
}

\email{scopatz@wisc.edu}

%%%% packages and definitions (optional)
\usepackage{color}
\usepackage{graphicx} % allows inclusion of graphics
\usepackage{booktabs} % nice rules (thick lines) for tables
\usepackage{microtype} % improves typography for PDF
\usepackage{xspace}
\usepackage{listings}
\usepackage{textcomp}
\usepackage{ulem}

\usepackage[usenames,dvipsnames]{xcolor}


\definecolor{listinggray}{gray}{0.9}
\definecolor{lbcolor}{rgb}{0.9,0.9,0.9}
\lstset{
    %backgroundcolor=\color{lbcolor},
    language={Python},
    tabsize=4,
    rulecolor=\color{black},
    upquote=true,
    aboveskip={1.5\baselineskip},
    belowskip={1.5\baselineskip},
    columns=fixed,
    extendedchars=true,
    breaklines=true,
    prebreak=\raisebox{0ex}[0ex][0ex]{\ensuremath{\hookleftarrow}},
    frame=single,
    showtabs=false,
    showspaces=false,
    showstringspaces=false,
    basicstyle=\scriptsize\ttfamily\color{green!40!black},
    keywordstyle=\color[rgb]{0,0,1.0},
    commentstyle=\color[rgb]{0.133,0.545,0.133},
    stringstyle=\color[rgb]{0.627,0.126,0.941},
    numberstyle=\color[rgb]{0,1,0},
    identifierstyle=\color{black},
    captionpos=t,
}

\renewcommand{\vec}[1]{\bm{#1}} %vector is bold italic
\newcommand{\vd}{\bm{\cdot}} % slightly bold vector dot
\newcommand{\grad}{\vec{\nabla}} % gradient
\newcommand{\ud}{\mathop{}\!\mathrm{d}} % upright derivative symbol
\newcommand{\cyclus}{\textsc{Cyclus}\xspace}
\newcommand{\Cyclus}{\cyclus}

\begin{document}
%%%%%%%%%%%%%%%%%%%%%%%%%%%%%%%%%%%%%%%%%%%%%%%%%%%%%%%%%%%%%%%%%%%%%%%%%%%%%%%%
\section{Introduction}

The Bateman equations governing radioactive decay are an important 
subexpression of generalized transmutation equations. In many cases, it is 
desirable to compute decay on its own, outside of the presence of a neutron 
or photon field.  In this case radioactive decay is a function solely on 
intrinsic physical parameters, namely half-lives. This document recasts the 
Bateman equations into a form that is better suited for computation than the 
traditional expression.

This use case is particularly relevant to the study of the nuclear fuel 
cycle (NFC). While reactors are a critical component of most NFCs, there 
are many places in the cycle where fresh and used fuel is sitting outside 
of a core. In these cases, decay alone is needed. Full transumation 
calculations require excessive computational effort when simply trying to 
model a storage facility. Thus, an efficient decay calculator has the 
potential to speed up simulation of the entire NFC.

Here, the traditional Bateman decay equations have been reformualted to 
a expression that is more suited for binary compuation. Coupled with a 
novel code generation technique, this binary formulation provides 
dramatic speed increases in a modern computing environment. 

Notably, all of the software implemented for the decay calculation is 
available as part of the free \& open source PyNE library 
\cite{pyne2014,Bates2014}. Furthermore, this decay functionality has
been fully integrated into the \cyclus fuel cycle simulator
\cite{cyclus2015,cyclus_v1.2}.

This paper will start by reviewing the classic version of the Bateman 
equations in \S II before providing the desired departure from 
this form in \S III. Following this, implementation specific 
approximations are discussed in \S IV.

%%%%%%%%%%%%%%%%%%%%%%%%%%%%%%%%%%%%%%%%%%%%%%%%%%%%%%%%%%%%%%%%%%%%%%%%%%%%%%%%
\section{Canonical Bateman Equations for Decay}
\label{canon}
The canonical expression of the Bateman equations for a decay chain 
proceeding from a nuclide $A$ to a nuclide $Z$ at time 
$t$ following a specific path is as follows \cite{Cetnar2006640}:

\begin{equation}
\label{bm-eq}
N_C(t) = \frac{N_1(0)}{\lambda_C} \cdot \gamma \cdot \sum_{i=1}^C \lambda_i c_{i} e^{-\lambda_i t}
\end{equation}

\begin{table}[hbt]
\label{decay-symbol-meaning}
\caption{Symbol Meaning in Deacy Equations}
\begin{tabular}{|l|l|}
\hline
\textbf{Symbol} & \textbf{Meaning} \\
\hline
$C$         & length of the decay chain\\
$i$         & index for ith species, on range [1, C]\\
$j$         & index for jth species, on range [1, C]\\
$t$         & time [seconds]\\
$N_i(t)$    & number density of the ith species at time t\\
$t_{1/2,i}$ & half-life of the ith species\\
$\lambda_i$ & decay constant of ith species, $ln(2)/t_{1/2,i}$\\
$\gamma$    & total branch ratio for this chain\\
\hline
\end{tabular}
\end{table}

The symbols in Equation \ref{bm-eq} have the meaning described in Table I.
Additionally, $c_{i}$ is defined as:
\begin{equation}
\label{c_i}
c_i = \prod_{j=1,i\ne j}^C \frac{\lambda_j}{\lambda_j - \lambda_i}
\end{equation}
Furthermore, the total chain branch ratio is defined as the product of the 
branch ratio between any two species in the chain \cite{harr2007precise}:

\begin{equation}
\label{gamma}
\gamma = \prod_{i=i}^{C-1} \gamma_{i \to i+1}
\end{equation}

Minor modifications are needed to Equation \ref{bm-eq} for terminal species: 
the first nuclide of a decay chain and the ending stable species. By setting 
$C=1$ for the first nuclide in a chain, the Bateman equations can be reduced 
to simply:

\begin{equation}
\label{N_C}
N_C(t) = N_1(0) e^{-\lambda_1 t}
\end{equation}

For stable species at the end of a chain, the appropriate equation is derived 
by taking the limit of when the decay constant of the stable nuclide 
($\lambda_C$) goes to zero.  Also notice that every $c_i$ contains exactly one $\lambda_C$
in the numerator which cancels with the $\lambda_C$ in the denominator 
in front of the summation:

\begin{equation}
\label{lim_lam}
\lim_{\lambda_C \to 0} N_C(t) = N_1(0)  \gamma \left[e^{-0t} + \sum_{i=1}^{C-1} \lambda_i \left(\frac{1}{0 - \lambda_i} \prod_{j=1,i\ne j}^{C-1} \frac{\lambda_j}{\lambda_j - \lambda_i} \right) e^{-\lambda_i t} \right]
\end{equation}

\begin{equation}
\label{lim_lam_N_C}
N_C(t) = N_1(0)  \gamma \left[1.0 - \sum_{i=1}^{C-1} \left(\prod_{j=1,i\ne j}^{C-1} \frac{\lambda_j}{\lambda_j - \lambda_i} \right) e^{-\lambda_i t} \right]
\end{equation}

Thus, Equations \ref{bm-eq}, \ref{N_C}, \& \ref{lim_lam_N_C} together are
able to compute the concetration of any species under decay 
at any point in time.

\section{Binary Reformulation of Bateman Equations}
\label{bin}
There are two main strategies may be used to construct a version of these 
equations that is better suited to computation, if not clarity. 

The first strategy is to minimize the number of arithmetic
operations that must be performed to achieve the same result. 
This can be done by grouping constants together and pre-calculating a single 
resultant constant. This saves the 
computer from having to perform these same arithmetic operations at run time.  
It is therefore possible to express the Bateman equations as a simple sum of 
exponentials, as seen in Equation \ref{N_C_bin}.

\begin{equation}
\label{N_C_bin}
N_C(t) = N_1(0) \sum_{i=1}^C k_{i} e^{-\lambda_i t}
\end{equation}

In the above equation, the coefficients $k_i$ are defined as follows:

\begin{equation}
\label{k_i}
k_i = \frac{\gamma}{\lambda_C} \lambda_i c_i
\end{equation}

If $k_i$ are computed at run time then this formualtion results in much 
more computational effort that than the original Bateman equations. This is 
because the $\gamma/\lambda_C$ term is brought into the top-level summation 
in Equation \ref{N_C_bin}. Thus increased the number of operations in the 
sum by $C$ to a total of $4C+1$, up from $3C+1$.
However, when the $k_i$s  are pre-caluclated, 
the number of artimetic operations for each term in the sum is a mere 4.
It is no longer a function of the length of the chain $C$ at all.  This is 
because the explicitly computing $c_i$ has been completely avoided.
Note that even for the shortest non-stable species, this computation is 
equivalently expensive as the normal Bateman equations, namely 4 operations.
Thus even in the worst case, the binary formulation performs just as well.

The second strategy is to note that computers are much better at dealing with 
powers of 2 then then any other base, even the natural base $e$. Thus the 
$\mathrm{exp2}(x)$ function, or $2^x$, is faster than the natural exponential 
function $\exp(x)$, $e^x$.  This is due to the number of floating point
operations needed to in the implementation of the underlying exponential 
functions. As proof of this principle, the following simple timing results
demonstrate that $\mathrm{exp2}(x)$ is faster for random data:

\begin{lstlisting}[caption={Exponential Timing Comparison}, 
                   label=expcmp]
In [1]: import numpy as np

In [2]: r = np.random.random(1000) / np.random.random(1000)

In [3]: %timeit np.exp(r)
10000 loops, best of 3: 26.6 us per loop

In [4]: %timeit np.exp2(r)
10000 loops, best of 3: 20.1 us per loop
\end{lstlisting}

From here, $\mathrm{exp2}(x)$ enables a savings of about 25\% over 
$\exp(x)$.  Since the core of the Bateman equations are exponentials, 
such a savings has the ability to reach all portions of a decay calculation.
Luckily, the decay constant itself provides an intrinsic mechanism to convert 
from base-e to base-2, as seen in the following equations:

\begin{equation}
\label{b2-0}
N_C(t) = N_1(0) \sum_{i=1}^C k_{i} \cdot e^{-\lambda_i t}
\end{equation}

\begin{equation}
\label{b2-1}
N_C(t) = N_1(0) \sum_{i=1}^C k_{i} \cdot \exp\left[\frac{-\ln(2)\cdot t}{t_{1/2,i}}\right]
\end{equation}

\begin{equation}
\label{b2-2}
N_C(t) = N_1(0) \sum_{i=1}^C k_{i} \cdot \mathrm{exp2}\left[\frac{-t}{t_{1/2,i}}\right]
\end{equation}

Equation \ref{b2-2} can be further collapsed by defining the constants $a$ to 
be the pre-computed exponent coefficient values:

\begin{equation}
\label{a_i}
a_i = \frac{-1}{t_{1/2,i}}
\end{equation}

Thus, the final form of the binary representation of the Bateman equations 
are shown in Equations \ref{nc_wakka}-\ref{nc_brown_shoes}:

\textbf{General Formulation:}

\begin{equation}
\label{nc_wakka}
N_C(t) = N_1(0) \sum_{i=1}^C k_{i} \cdot 2^{a_i t}
\end{equation}

\textbf{First Nuclide in Chain:}

\begin{equation}
\label{nc_jawakka}
N_C(t) = N_1(0) \cdot 2^{a_1 t}
\end{equation}

\textbf{Stable Nuclide:}

\begin{equation}
\label{nc_brown_shoes}
N_C(t) = N_1(0) \left[1.0 + \sum_{i=1}^{C-1} \lim_{\lambda_C\to 0}(k_{i}) \cdot 2^{a_i t} \right]
\end{equation}

With completely precomputed $k$, $a$, and the $\mathrm{exp2}()$ function, this 
formulation reduces the number of arithmetic operations to 3 for each term 
in the chain. Furthermore, it completely preserves all physical processes 
because so assumptions were made aside from the Bateman equations themselves.
Note that it is not possible to reduce the number of operations further
without assuming a value for $t$. Thus this reperesentation is the 
mathematically fastest formulation of the Bateman equations.

\section{Implementation Specific Approximations}
\label{approx}
The method presented in the previous section hase been implemented within
PyNE and Cyclus. However, any given mathematical system of equations
will encounter implementation-specific choices when turned into 
associated software. Thus, though the above formulation holds generally for 
any decay chain, certain approximations have been used in the implementation
for reasons of perfomance, compile-ability, and computability.
The PyNE implementation generally aims to reduce the number of 
chains and terms that are calculated when they can be shown to be 
redundant or insignuficant to the total calculation. In specific, the 
following modeling assumptions have been made:

\begin{enumerate}
\item Decay chains coming from spontaneous fission are not tallied as they 
    lead to an explosion of the total number of chains while contributing to 
    extraordinarily rare branches.
\item Decay alphas are not treated as He-4 production.
\item For chains longer than length 2, any 
    term whose half-life is less than $10^{-8}$ of the sum of all 
    half-lives in the chain is dropped. This filtering prevents excessive
    calculation from species which do not significantly contribute to 
    end atom fractions. The relative threshold $10^{-8}$ was chosen
    because it is a reasonable naive estimate of floating point error after 
    many operations. If the filtering causes there to be less than 
    two terms in the summation, then the filtering is turned off and all
    terms are computed.
\item To prevent other sources of floating point error, a nuclide is 
    determined to be stable when $\lambda_i < 10^{-16}$, rather than when 
    $\lambda_i = 0.0$.
\item If a chain has any $\mathrm{NaN}$ decay constants, the chain in rejected.
\item If a chain has any infinite $k$ values, the chain in rejected.
\end{enumerate}

In principle, each of these statements is reasonable. However, they may 
have covariant effects that inject unreasonable errors into the system.
To ensure that this is not the case for the assumptions above, 
a benchmark study was performed \cite{benchmark}. This benchmark compared
the results of running ORIGEN v2.2 \cite{croff1980origen2} to the results of 
the binary formulation. This study compares 2781 cases for a wide spread of
nuclides over 0.01, 0.1, 1.0, 10, 100 times the half-life of the nulcide.
In the overwhemling majority of cases, the relative errors were found to be 
within acceptable bounds.  For cases that did prove significantly different, 
it differences could all be traced back to discrepencies in the underling 
data. Since the PyNE binary formulation uses more recent ENSDF decay 
data \cite{bhat1992evaluated} and computes more decay chains than ORIGEN, 
it is reasonble to assume that these discrepencies are not the fault of the 
binary formulation or the implementation assumptions.

It is also important to mention that these implmentation assumptions 
may preclude desired behavior by users. For example, perhaps alpha decay 
should be considered He-4 production. This would bring the implementation in
line with the same feature in ORIGEN v2.2. In such a situations, these 
assumptions should be revisited and the implmentation should be reworked.

%%%%%%%%%%%%%%%%%%%%%%%%%%%%%%%%%%%%%%%%%%%%%%%%%%%%%%%%%%%%%%%%%%%%%%%%%%%%%%%%
\section{Conclusions \& Future Work}

%%%%%%%%%%%%%%%%%%%%%%%%%%%%%%%%%%%%%%%%%%%%%%%%%%%%%%%%%%%%%%%%%%%%%%%%%%%%%%%%
%\section{Acknowledgments}

%%%%%%%%%%%%%%%%%%%%%%%%%%%%%%%%%%%%%%%%%%%%%%%%%%%%%%%%%%%%%%%%%%%%%%%%%%%%%%%%
\bibliographystyle{ans}
\bibliography{bibliography}
\end{document}
